%%%%%%%% Padrão de Relatório 2017 %%%%%%%%
% Esse padrão de relatório foi baseado na norma ABNT NBR 10719:2015, para Relatórios técnicos e/ou científicos
%%%%%%%%%%%%%%%%%%%%%%%%%%%%%%%%%%%%%%%%%%
\documentclass[12pt,openany,oneside,a4paper,portuguese,brazil]{abntex2}
\usepackage[utf8]{inputenc}	
\usepackage{listingsutf8}
\usepackage{amsmath}
\usepackage{tikz}
\usepackage{tocloft}
\usetikzlibrary{shapes.geometric, arrows}

\usepackage[T1]{fontenc}		           % Seleção de códigos de fonte.

\usepackage{lmodern} 

% Conversão automática dos acentos
\usepackage[alf]{abntex2cite}
\usepackage{indentfirst}		           % Indenta o primeiro parágrafo de cada seção.
\renewcommand\thesection{\arabic{section}} % Reestabelece o nível da categoria "seção"
\usepackage{graphicx}			           % Inclusão de gráficos
\usepackage{amsmath}
\usepackage{microtype} 	                   % Para melhorias de justificação
\usepackage{multicol}                      % Para mesclar células (coluna)
\usepackage{multirow}                      % Para mesclar células (linha)
\usepackage{float}                         % Mais opções de posicionamento [H]
\usepackage{xcolor}  %Opções de cor

\setlrmarginsandblock{2.5cm}{2cm}{*}     % Recomendado pelo Regulamento
\setulmarginsandblock{2cm}{2cm}{*}     % Recomendado pelo Regulamento
\checkandfixthelayout
%%%%%%%%%%%%%%%%%%%%%%%%%%%%%%%%%%%%%%%%%%
%Pacotes e configurações adicionais:
\usepackage{tikz}                          % Recursos para gráficos, figuras, etc.
\usetikzlibrary{arrows.meta,arrows,shapes,positioning,shadows,trees}               
\usetikzlibrary{backgrounds}               % Usado para adicionar boxes nas figuras
\usepackage{subfig}                        % Utilizado para colocar figuras lado a lado
\usepackage{pgfplots}                      % Utilizado para plotagem de gráficos
\usepackage{array}
\tikzstyle{startstop} = [rectangle, rounded corners, minimum width=3cm, minimum height=1cm, text centered, draw=black, fill=red!30]
\tikzstyle{process} = [rectangle, minimum width=3cm, minimum height=1cm, text centered, draw=black, fill=blue!20]
\tikzstyle{decision} = [diamond, minimum width=3cm, minimum height=1cm, text centered, draw=black, fill=yellow!30]
\tikzstyle{arrow} = [thick,->,>=stealth]

\pgfplotsset{width=0.45\textwidth,         % Largura padrão dos gráficos
             compat=1.13}                  % Versão do pacote
\pgfplotsset{axis line style={black},
             every x tick label/.append style={font=\small, yshift=0.5ex},
             every y tick label/.append style={font=\small, xshift=0.5ex}}
\captionsetup{font=normalsize,labelfont={bf,sf}}

%\captionsetup[subfigure]{font=normalsize,labelfont={bf,sf}}
\usepackage{color}				           % Controle das cores
\usepackage{gensymb}                       % Utilizado para inserir símbolos como °
\usepackage{amsmath}                       % Equações matemáticas melhoradas
\usepackage{hyperref}
\usepackage{url}
\hypersetup{
    breaklinks=true
}
% Para inserir URLs no texto

\usepackage{listings}
\usepackage{xcolor}

\hypersetup{                               % Altera as cores dos links do documento PDF
    colorlinks,
    linkcolor={blue!50!black},
    citecolor={blue!50!black},
    urlcolor={blue!80!black}
}
\numberwithin{equation}{section}                            % Adiciona o número da seção às equações
\addto\captionsbrazil{\renewcommand{\contentsname}{Índice}} %Altera o nome de "Sumário" para "Índice", como no Regulamento
\usepackage{lastpage}                                       % Utilizado para numerar as páginas
\makepagestyle{mecsol}
  \makeoddhead{mecsol}
     {}
     {}
     {\thepage}
%%%%%%%%%%%%%%%%%%%%%%%%%%%%%%%%%%%%%%%%%%
%%%%%%%%%%%%%%%%%%%%%%%%%%%%%%%%%%%%%%%%%%
%Pacotes e configurações adicionais:
%\usepackage{tikz}                          % Recursos para gráficos, figuras, etc.
%\usetikzlibrary{arrows.meta,arrows,shapes,positioning,shadows,trees}               
%\usetikzlibrary{backgrounds}               % Usado para adicionar boxes nas figuras
\usepackage{titlesec}

% Definir formato da seção
\titleformat{\section}{\normalfont\bfseries}{\thesection}{1em}{}

% Definir formato da subseção com mais recuo
\titleformat{\subsection}{\normalfont\bfseries}{\thesubsection}{1em}{}
%%%%%%%%%%%%%%%%%%%%%%%%%%%%%%%%%%%%%%%%%
\titlespacing* % starred version: first paragraph is not indented
    {\subsection} % <command>
    {2em} % <left>
    {3.5ex plus 1ex minus .2ex} % <before-sep>
    {2.3ex plus .2ex} % <after-sep>

\addto\captionsbrazil{\renewcommand{\contentsname}{Sumário}}
\renewcommand{\normalfont}{\bfseries\rmfamily} % Fonte normal em negrito e roman
\renewcommand{\cftsectionfont}{\bfseries\rmfamily} % Seções em negrito e roman
\renewcommand{\cftsubsectionfont}{\rmfamily} % Subseções em negrito, itálico e roman
\renewcommand{\cftsubsubsectionfont}{\rmfamily} % Subsubseções em negrito e roman


\usepackage{hyperref} % Para links clicáveis no sumário
\hypersetup{
    colorlinks=true,
    linkcolor=black,  % Links internos (para o sumário, por exemplo)
    citecolor=black,  % Links para citações
    urlcolor=black    % Links de URLs
}


% Definir o estilo do código
\lstset{
    inputencoding=utf8,             % Configura o input como UTF-8 para o código
    basicstyle=\ttfamily\small,     % Usa fonte monoespaçada (teletipo)
    breaklines=true,                % Quebra as linhas automaticamente
    rulecolor=\color{black}, 
    frame=single,                   % Coloca uma borda ao redor do código
    numbers=left,                   % Mostra números de linha à esquerda
    numberstyle=\tiny,              % Estilo dos números de linha (sem cor)
    keywordstyle=\color{blue},      % Cor para palavras-chave (como if, while)
    commentstyle=\color{gray},      % Cor para comentários
    stringstyle=\color{red},        % Cor para strings
    showstringspaces=false,         % Não mostra espaços em branco
    escapeinside={(*@}{@*)},        % Para usar comandos LaTeX dentro do código
    tabsize=2                       % Tamanho do tab (alinhamento)
}


\begin{document}

\begin{center}

\textbf{UNIVERSIDADE FEDERAL DE SANTA CATARINA}\\[0.1cm] 
\textbf{DEPARTAMENTO DE INFORMÁTICA E ESTATÍSTICA}\\[0.1cm] 
\textbf{CURSO DE GRADUAÇÃO EM CIÊNCIAS DA COMPUTAÇÃO}\\[0.1cm] 
\textbf{INE5408 - ESTRUTURAS DE DADOS}\\[4cm]

\text{\large LEONARDO DE SOUSA MARQUES}\\ 
\vfill

\textsc{\large \bfseries PROJETO I: \\ Verificação de cenários e determinação de área limpa por um robô aspirador}\\[3cm] 
\end{center}

\noindent
\textbf{\large Docente:} \\[0.2cm]
\text{\large Dr. Alexandre Gonçalves Silva} 


\vfill
\begin{center}

\text{Florianópolis}\\ 
\text{2024}\\[1cm] 
    
\end{center}

\pagebreak

\pagebreak
\textual
\pagestyle{mecsol}
\aliaspagestyle{chapter}{mecsol_capitulo}% customizing chapter pagestyle
% Sumário
\tableofcontents
\newpage

% Lista de Tabelas
\listoftables

\pagebreak
\section{INTRODUÇÃO}

O seguinte relatório tem como objetivo analisar e resolver dois problemas relacionados a um arquivo XML (\textit{Extensible Markup Language}). No primeiro, vamos verificar se o aninhamento das chaves (< e >) e fechamento das marcações (tags) está correto. Já no segundo, vamos calcular a área total que um robô pode limpar partindo de uma posição (x, y) em cada uma das matrizes fornecidas nos arquivos.

\subsection{Modelo de Arquivo}

Antes de começar, vamos entender o modelo do arquivo XML, conforme o bloco abaixo.

\begin{lstlisting}[language=XML]
<cenarios>
<cenario>
<nome>cenario-01</nome>
<dimensoes><altura>20</altura><largura>30</largura></dimensoes>
<robo><x>10</x><y>20</y></robo>
<matriz>
000000000000000000000000000000
000000000000000000000000000000
000000000000000000000000000000
000000000000000000000000000000
000000000000000000000000000000
000000000000000000000000000000
000000000001110000000000000000
000000000011000000000000000000
001100100111110011111000111100
001100100011000011000001110000
001100100011000011111001100000
001100100011000000111101100000
001101100011000010011101110000
000111100011000011111000111100
000000000000000000000000000000
000000000000000000000000000000
000000000000000000000000000000
000000000000000000000000000000
000000000000000000000000000000
000000000000000000000000000000
</matriz>
</cenario>
... % Continue com os outros cenarios da mesma forma
</cenarios>
\end{lstlisting}

Com isso, conseguimos extrair as informações de que o robô começará na posição P, tal que P(x, y) = (10, 20), além de que altura = 20 e largura = 30, por exemplo. Essas informações serão úteis para a questão 2. Em relação aos aninhamentos, percebemos que para o trecho mostrado, ele se demonstra correto. Tendo visto o modelo dos arquivos que estamos trabalhando, podemos dar início a resolução dos problemas apresentados.

\section{PROBLEMA 1}

Para verificar os aninhamentos de uma string que representa todo o arquivo XML lido, é necessário realizar essa conversão. Por isso, o docente forneceu o seguinte bloco de código, na função main, que lê o conteúdo do arquivo com nome passado por linha de comando no terminal - ex: \texttt{cenarios1.xml}.


\begin{lstlisting}[language=c++]
int main() {

    string filename;
    std::cin >> filename;

    // Abertura do arquivo
    ifstream filexml(filename);
    if (!filexml.is_open()) {
        cerr << "Erro ao abrir o arquivo " << filename << endl;
        throw runtime_error("Erro no arquivo XML");
    }

    // Leitura do XML completo para 'texto'
    string texto;
    char character;
    while (filexml.get(character)) {
        texto += character;
    }

    ...
\end{lstlisting}

Tendo a string texto, podemos criar a função \texttt{bool verifica\_aninhamentos(string texto)}, que retorna um valor booleano (verdadeiro ou falso). Para implementar essa função, vamos utilizar a estrutura de dados linear de \textbf{Pilha} (\textit{Array Stack}), que segue a lógica LIFO (\textit{Last in, first out}), que vai armazenar as tags que estão sendo abertas e fechadas. (O código completo da pilha está descrito no Anexo 1.)

Seguindo essa lógica, vamos precisar do comprimento da string para poder instanciar uma pilha com esse tamanho - note que poderíamos utilizar um valor menor, visto que o arquivo não é composto apenas por tags, mas, dessa forma, anunciamos o tamanho máximo. Assim, a lógica para resolver o problema consistiu em percorrer a string texto até o momento em que um caracter '<' é encontrado e, a partir disso, escrever a tag que ele abre até o fechamento da mesma no caractere '>'. O motivo de escrevermos a tag inteira é para caso o caractere seguinte a '<' for uma barra de fechamento '/', pois, dessa forma, garantimos que todas estão sendo devidamente validadas.
\begin{lstlisting}[language=c++][utf-8]
bool verifica_aninhamentos(string texto) {
    // Calcula o comprimento da string de entrada
    int len = texto.length();
    // Cria uma pilha para armazenar as tags de abertura
    structures::ArrayStack<string> stack_tags(len);
    // Inicializa uma string vazia para armazenar o nome da tag atual
    string tag = "";
    // Percorre cada caractere da string de entrada
    for (int i = 0; i < len; i++) {
        // Se o caractere atual for um '<', significa que uma tag vai ser escrita
        if (texto[i] == '<') {
            // Verifica se e uma tag de fechamento se o proximo caractere for '/'
            bool is_closing_tag = (i + 1 < len &&
                                   texto[i + 1] == '/');
            // Se for uma tag de fechamento, avanca o indice para pular o '/'
            if (is_closing_tag) {
                i++;
            }
            // Percorre a string ate encontrar o '>' para capturar o nome da tag
            for (int j = i + 1; j < len; j++) {
                // Se o caractere atual for '>', a tag foi completamente capturada
                if (texto[j] == '>') {
                    // Atualiza 'i' para a posicao de '>', indicando o fim da tag
                    i = j;
                    // Sai do loop interno
                    break;
                }
                // Adiciona o caractere atual ao nome da tag
                tag += texto[j];
            }
            
            // Se for uma tag de fechamento, verifica se a tag de abertura esta no topo da pilha
            if (is_closing_tag) {
                // Se a pilha estiver vazia ou a tag de abertura no topo da pilha nao corresponder a tag de fechamento, retorna falso
                if (stack_tags.empty() || stack_tags.pop() != tag) {
                    return false;
                }
            } else {
                // Se for uma tag de abertura, empilha o nome da tag
                stack_tags.push(tag);
            }
        }
        // Inicializa a variavel tag como vazia para a proxima
        tag = "";
    }
    // Se a pilha estiver vazia, significa que todas as tags foram fechadas
    return stack_tags.empty();
}
\end{lstlisting}

\subsection{Exemplo de Funcionamento}
Vamos observar o exemplo da verificação de aninhamentos, com as tags sendo empilhadas e desempilhadas à medida que são analisadas. Considere o código XML:
\begin{lstlisting}[language=XML]
<cenarios>
  <cenario>
    <nome>cenario-01</nome>
  </cenario>
</cenarios>
\end{lstlisting}

Com isso, podemos esquematizar a tabela que representa as etapas de empilhamento para verificar os passos dos algoritmo.

\newpage

\begin{table}[h!]
\centering
\begin{tabular}{|c|c|}
\hline
\textbf{Etapa} & \textbf{Pilha} \\
\hline
1. Início & \(\emptyset\) \\
\hline
2. Tag \texttt{<cenarios>} encontrada & \texttt{<cenarios>} \\
\hline
3. Tag \texttt{<cenario>} encontrada & \texttt{<cenario>} \\
                                      & \texttt{<cenarios>} \\
\hline
4. Tag \texttt{<nome>} encontrada & \texttt{<nome>} \\
                                   & \texttt{<cenario>} \\
                                   & \texttt{<cenarios>} \\
\hline
5. Tag \texttt{</nome>} encontrada e removida & \texttt{<cenario>} \\
                                              & \texttt{<cenarios>} \\
\hline
6. Tag \texttt{</cenario>} encontrada e removida & \texttt{<cenarios>} \\
\hline
7. Tag \texttt{</cenarios>} encontrada e removida & \(\emptyset\) \\
\hline
\end{tabular}
\caption{Exemplo de verificação de aninhamento}
\end{table}

A pilha está vazia no final (\(\emptyset\)), o que indica que todas as tags de abertura tiveram uma correspondente tag de fechamento, confirmando que o aninhamento está correto.

Também podemos verificar um caso em que tenhamos erros de aninhamento. 

\begin{lstlisting}[language=XML]
<cenarios>
  <cenario>
    <nome>cenario-01<nome> -- erro;
  </cenario>
</cenarios>
\end{lstlisting}

\begin{table}[h!]
\centering
\begin{tabular}{|c|c|}
\hline
\textbf{Etapa} & \textbf{Estado da Pilha} \\
\hline
1. Início & \(\emptyset\) \\
\hline
2. Tag \texttt{<cenarios>} encontrada & \texttt{<cenarios>} \\
\hline
3. Tag \texttt{<cenario>} encontrada & \texttt{<cenario>} \\
                                      & \texttt{<cenarios>} \\
\hline
4. Tag \texttt{<nome>} encontrada & \texttt{<nome>} \\
                                        & \texttt{<cenario>} \\
                                        & \texttt{<cenarios>} \\
\hline
5. Tag \texttt{<nome>} encontrada & \texttt{<nome>} \\
                                     & \texttt{<nome>} \\
                                     & \texttt{<cenario>} \\
                                     & \texttt{<cenarios>} \\
\hline
6. Tag \texttt{</cenario>} encontrada & \texttt{<nome>} \\
                                     & \texttt{<nome>} \\
                                     & \texttt{<dimensoes>} \\
                                     & \texttt{<cenario>} \\
                                     & \texttt{<cenarios>} \\
\hline
\end{tabular}
\caption{Processo de Verificação com Erro no Aninhamento}
\end{table}

No passo 6 o método irá retornar $false$, pois \texttt{<cenario>} não está no topo da pilha, visto que houve um erro durante o processo. Nesse caso, como a tag \texttt{<nome>} não foi fechada, ela impede o fechamento das anteriores.


   
\pagebreak      

\section{PROBLEMA 2}

Neste problema, o objetivo é calcular a área total que um robô consegue limpar, começando de uma posição inicial (x, y) dentro de uma matriz binária. Cada matriz no arquivo representa um ambiente, e as células com valor '1' indicam áreas que o robô pode limpar, enquanto as células com '0' representam obstáculos ou áreas já limpas.


O algoritmo de limpeza segue uma lógica de propagação a partir da posição inicial (x, y). A partir dessa posição, o robô pode se mover em quatro direções diferentes, que são descritas pelo vetor de direções $\vec{d}$  = $\{(-1, 0), (1, 0), (0, -1), (0, 1)\}$. Essas direções correspondem, respectivamente, aos \textbf{movimentos para cima, para baixo, para a esquerda e para a direita}. A cada movimento, as coordenadas (x, y) do robô são atualizadas somando-se os valores dessas direções ao valor atual da posição, permitindo que ele explore áreas adjacentes.


O robô continuará se movendo para todas as células adjacentes que contêm o valor '1' e que ainda não foram visitadas, até que todas as possíveis áreas acessíveis a partir da posição inicial sejam exploradas. Ao final do processo, a \textbf{área} total limpa pelo robô será a soma de todas as células conectadas que contêm '1'.

Para isso, alguns códigos para a extração das informações relevantes dos arquivos XML já foram fornecidos pelo docente. Nesse caso, temos a \textbf{classe Cenário}, com atributos públicos size\_t \textbf{nome}, size\_t \textbf{altura}, size\_t \textbf{largura}, size\_t \textbf{x}, size\_t \textbf{y}, string \textbf{matriz} e size\_t \textbf{índice final} - este último representa o índice final do cenário lido. Vale ressaltar também que o construtor dessa classe recebe como parâmetros a string texto e o índice inicial de leitura do cenário: \textbf{Cenario(string& texto, size\_t indice\_inicial)}. Portanto, para o primeiro caso, vamos inicilizar com indice\_inicial = 0;

Nota-se, também, que a matriz está sendo dada, na verdade, em forma de string. Então, para acessarmos um índice correspondente à linha $i$ e coluna $j$, precisaremos realizar o cálculo $E[i \times \text{largura} + j]$, onde E é a matriz string e $0 \leq i < \text{largura}$ e $0 \leq j < \text{altura}$.

Sabendo disso, primeiramente criamos um método para percorrer a string texto e retornar a quantidade de matrizes presentes no cenário. Isso será útil para atualizar a lógica de cenários que necessita de um índice inicial na instanciação, como vimos anteriormente.
\begin{lstlisting}[language=c++]
size_t quantidade_matrizes(string texto) {
    size_t count = 0;
    size_t pos = 0;
    while ((pos = texto.find("<matriz>", pos)) != string::npos) {
        count++;
        pos += 8; // comprimento da tag <matriz>
    }
    return count;
}
\end{lstlisting}

Agora, precisamos criar de fato a função que recebe um cenário e extrai a área a ser limpa pelo robô, chamada \texttt{int calcula\_area(Cenario\& c)}. Nesse mesmo método, iremos utilizar uma estrutura de dados de Fila (\textit{Array Queue}), que segue a lógica FIFO (\textit{First in, first out}), para armazenar os pares (x, y) acessíveis. (O código completo da fila está descrito no Anexo 2.)
\begin{lstlisting}[language=c++]
size_t calcula_area(Cenario& c) {

    // Copia a matriz, posicao inicial do robo e as dimensoes do cenario
    string E = c.matriz;
    size_t x = c.x;
    size_t y = c.y;
    size_t altura = c.altura;
    size_t largura = c.largura;

    // Inicializa a variavel de area como 0 (nao ha area limpa inicialmente)
    size_t area = 0;

    // Vetor de direcoes movimentos possiveis (cima, baixo, esquerda e direita)
    vector<pair<size_t, size_t>> direcoes = {{-1, 0}, {1, 0}, 
                                             {0, -1}, {0, 1}};
    
    // Fila para explorar as posicoes acessiveis
    structures::ArrayQueue<pair<size_t, size_t>> queue(altura*largura);

    // Vetor de controle que marca as posicoes ja visitadas na matriz
    string R = "";
    for (int i = 0; i < E.length(); i++) {
        R += '0';  // Inicializa todas as posicoes como nao visitadas
    }

    // Verifica se a posicao inicial (x, y) e uma celula acessivel ('1')
    if (E[x * largura + y] == '1') {
        // Marca a posicao como visitada e adiciona a fila de exploracao
        R[x * largura + y] = '1';
        queue.enqueue(make_pair(x, y));
        area++;  // Incrementa a area limpa
    }

    // Loop para processar todas as celulas acessiveis
    while (!queue.empty()) {
        // Desenfileira a proxima posicao a ser processada
        pair<size_t, size_t> aux = queue.dequeue();
        
        // Tenta explorar as 4 direcoes
        for (int i = 0; i < direcoes.size(); i++) {
            // Novas posicoes de x e y
            size_t nx = aux.first + direcoes[i].first;
            size_t ny = aux.second + direcoes[i].second;  

            // Verifica se a nova posicao esta dentro dos limites da matriz
            if (nx < altura && ny < largura) {
                size_t idx = nx * largura + ny; 
                // Se a nova posicao for acessivel ('1') e ainda nao foi visitada
                if (E[idx] == '1' && R[idx] == '0') {
                    // Enfileira a nova posicao e marca como visitada
                    queue.enqueue(make_pair(nx, ny));
                    R[idx] = '1';
                    area++;  // Incrementa a area limpa
                }
            }
        }
    }
    return area;
}
\end{lstlisting}

\subsection{Exemplo de Funcionamento}

Vamos exemplificar o funcionamento do algoritmo em uma matriz pequena para o problema de cálculo de área, presente no código XML seguinte.
\newpage
\begin{lstlisting}
<cenarios>
  <cenario>
    <nome>cenario-01</nome>
    <dimensoes><altura>5</altura><largura>5</largura></dimensoes>
    <robo><x>2</x><y>2</y></robo>
    <matriz>
      11100
      10100
      10111
      10001
      11111
    </matriz>
  </cenario>
</cenarios>
\end{lstlisting}

Para esse exemplo, temos que a posição inicial do robô será no ponto (2, 2). A tabela abaixo ilustra o processo de cálculo da área que o robô pode limpar, partindo da posição inicial. A coluna à esquerda mostra a etapa, e a coluna à direita mostra o estado da fila e as posições sendo processadas.

\begin{table}[h!]
\centering
\begin{tabular}{|c|c|}
\hline
\textbf{Etapa} & \textbf{Estado da Fila} \\
\hline
1. Início & (2, 2) \\
\hline
2. Posição (2, 2) processada, posições vizinhas enfileiradas: & (1, 2), (2, 3) \\
\hline
3. Posição (1, 2) processada, novas posições enfileiradas: & (2, 3), (0, 2) \\
\hline
4. Posição (2, 3) processada, novas posições enfileiradas: & (0, 2), (2, 4) \\
\hline
5. Posição (0, 2) processada, novas posições enfileiradas: & (2, 4), (0, 1) \\
\hline
6. Posição (2, 4) processada, novas posições enfileiradas: & (0, 1), (3, 4) \\
\hline
7. Posição (0, 1) processada, novas posições enfileiradas: & (3, 4), (0, 0) \\
\hline
8. Posição (3, 4) processada, novas posições enfileiradas: & (0, 0), (4, 4) \\
\hline
9. Posição (0, 0) processada, novas posições enfileiradas: & (4, 4), (1, 0) \\
\hline
10. Posição (4, 4) processada, novas posições enfileiradas: & (1, 0), (4, 3) \\
\hline
11. Posição (1, 0) processada, novas posições enfileiradas: & (4, 3), (2, 0) \\
\hline
12. Posição (4, 3) processada, novas posições enfileiradas: & (2, 0), (4, 2) \\
\hline
13. Posição (2, 0) processada, novas posições enfileiradas: & (4, 2), (3, 0) \\
\hline
14. Posição (4, 2) processada, novas posições enfileiradas: & (3, 0), (4, 1) \\
\hline
15. Posição (3, 0) processada, novas posições enfileiradas: & (4, 1), (4, 0) \\
\hline
16. Posição (4, 1) processada, sem novas posições: & (4, 0) \\
\hline
17. Posição (4, 0) processada, sem novas posições: & \emptyset\ \\
\hline
\end{tabular}
\caption{Processo de Cálculo de Área Usando Fila}
\end{table}

\pagebreak
\section{ARQUIVO MAIN}

Tendo criado os métodos que resolvem os problemas, precisamos chamá-los na função main(). É importante ressaltar que para o segundo problema, como podemos ter diversas matrizes no mesmo XML, é necessário fazer um loop que instância um $Cenario$ com posição inicial referente à posição final do último cenário lido. Desse modo, teremos a função main completa.

\begin{lstlisting}[language=c++]
int main() {

    string filename;
    cin >> filename;

    ifstream filexml(filename);
    if (!filexml.is_open()) {
        cerr << "Erro ao abrir o arquivo " << filename << endl;
        throw runtime_error("Erro no arquivo XML");
    }

    // Leitura do XML completo para 'texto'
    string texto;
    char character;
    while (filexml.get(character)) {
        texto += character;
    }

    // Problema 1
    bool aninhamentos = verifica_aninhamentos(texto);

    if (!aninhamentos) {
        cout << "Erro de aninhamentos" << endl;
    } else {
        // Problema 2
         size_t num_matrizes = quantidade_matrizes(texto);
         size_t i = 0;
         for (size_t m = 0; m < num_matrizes; m++) {
            Cenario c(texto, i); 
            size_t area = calcula_area(c);
            cout << c.nome << " " << area << endl;
            i = c.indice_final;
        }
    }
    return 0;
}
\end{lstlisting}
\pagebreak
\section{CONCLUSÃO}

Como conclusão, vamos apontar as dificuldades na resolução dos problemas. 

Em relação ao primeiro, o principal desafio se deu pelo fato de que não podíamos tratar como um simples problema de aninhamentos de abertura e fechamento de chaves '<' e '>'. Por isso, a solução adotada buscou implementar uma pilha que armazenava as tags completas. Assim, quando a sequência '</' fosse identificada, era necessário verificar se a string formada até o fechamento da chave correspondia à que estava no topo da pilha, garantindo que os pares de abertura e fechamento fossem corretos. Essa abordagem se mostrou eficaz para lidar com a estrutura de aninhamento mais complexa, além de verificar possíveis erros estruturais nos arquivos XML fornecidos.

Já no segundo problema, a estrutura de fila permitiu que o processamento das células visitadas fosse realizado de forma simples. Dessa forma, o uso da fila, em conjunto com a busca em largura, facilitou o processo de expansão da área de limpeza, mantendo o controle das posições já processadas e evitando que o robô saísse dos limites da string matriz.
\pagebreak
\section{REFERÊNCIAS}

\noindent JOYANES AGUILAR, Luis. \textbf{Programação em C++: algoritmos, estruturas de dados e objetos}. São Paulo:
McGraw Hill, 2008. xxxi, 768 p. ISBN 9788586804816.

\vspace{2em} % Isso adiciona uma linha de espaço

\noindent PANDEY, Himanshu. \textbf{Stack and Queue}. University of Lucknow. Disponível em: <\href{https://www.lkouniv.ac.in/site/writereaddata/siteContent/202003251324427324himanshu_Stack_Queue.pdf}{link}>. Acesso em 12 de outubro de 2024.
\pagebreak
\appendix{ANEXO 1 - ArrayStack.h}\\[0.2cm]

\begin{lstlisting}[language=c++]
// Copyright [2024] <LEONARDO DE SOUSA MARQUES>
#ifndef STRUCTURES_ARRAY_STACK_H
#define STRUCTURES_ARRAY_STACK_H

#include <cstdint>  // std::size_t
#include <stdexcept>  // C++ exceptions

namespace structures {

template<typename T>
//! CLASSE PILHA
class ArrayStack {
 public:
    //! construtor simples
    ArrayStack();
    //! construtor com parametro tamanho
    explicit ArrayStack(std::size_t max);
    //! destrutor
    ~ArrayStack();
    //! metodo empilha
    void push(const T& data);
    //! metodo desempilha
    T pop();
    //! metodo retorna o topo
    T& top();
    //! metodo limpa pilha
    void clear();
    //! metodo retorna tamanho
    std::size_t size();
    //! metodo retorna capacidade maxima
    std::size_t max_size();
    //! verifica se esta vazia
    bool empty();
    //! verifica se esta cheia
    bool full();

 private:
    T* contents;
    int top_;
    std::size_t max_size_;

    static const auto DEFAULT_SIZE = 10u;
};

}  // namespace structures

#endif


template<typename T>
structures::ArrayStack<T>::ArrayStack() {
    max_size_ = DEFAULT_SIZE;
    contents = new T[max_size_];
    top_ = -1;
}

template<typename T>
structures::ArrayStack<T>::ArrayStack(std::size_t max) {
    max_size_ = max;
    contents = new T[max];
    top_ = -1;
}

template<typename T>
structures::ArrayStack<T>::~ArrayStack() {
    delete [] contents;
}

template<typename T>
void structures::ArrayStack<T>::push(const T& data) {
    if (full()) {
        throw std::out_of_range("pilha cheia");
    } else {
        top_ += 1;
        contents[top_] = data;
    }
}

template<typename T>
T structures::ArrayStack<T>::pop() {
    if (empty()) {
        throw std::out_of_range("pilha vazia");
    }
    T beforePop = contents[top_];
    top_ -= 1;
    return beforePop;
}

template<typename T>
T& structures::ArrayStack<T>::top() {
    return contents[top_];
}

template<typename T>
void structures::ArrayStack<T>::clear() {
    top_ = -1;
}

template<typename T>
std::size_t structures::ArrayStack<T>::size() {
    return top_ + 1;
}

template<typename T>
std::size_t structures::ArrayStack<T>::max_size() {
    return max_size_;
}

template<typename T>
bool structures::ArrayStack<T>::empty() {
    return (top_ == -1);
}

template<typename T>
bool structures::ArrayStack<T>::full() {
    int max = static_cast<int> (max_size());
    return ( max == top_ + 1);
}
\end{lstlisting}
\pagebreak      
\appendix{ANEXO 2 - ArrayQueue.h}\\[0.2cm]

\begin{lstlisting}[language=c++]
// Copyright [2024] <LEONARDO DE SOUSA MARQUES>
#include <cstdint>  // std::size_t
#include <stdexcept>  // C++ Exceptions

namespace structures {
template<typename T>
//! classe ArrayQueue
class ArrayQueue {
 public:
    //! construtor padrao
    ArrayQueue();
    //! construtor com parametro
    explicit ArrayQueue(std::size_t max);
    //! destrutor padrao
    ~ArrayQueue();
    //! metodo enfileirar
    void enqueue(const T& data);
    //! metodo desenfileirar
    T dequeue();
    //! metodo retorna o ultimo
    T& back();
    //! metodo limpa a fila
    void clear();
    //! metodo retorna tamanho atual
    std::size_t size();
    //! metodo retorna tamanho maximo
    std::size_t max_size();
    //! metodo verifica se vazio
    bool empty();
    //! metodo verifica se esta cheio
    bool full();

 private:
    T* contents;
    std::size_t size_;  // tamanho atual da fila
    std::size_t max_size_;  // tamanho maximo que a fila pode ter
    int begin_;  // indice do inicio (para fila circular)
    int end_;  // indice do fim (para fila circular)
    static const auto DEFAULT_SIZE = 10u;
};

}  // namespace structures

//! construtor padrao
template<typename T>
structures::ArrayQueue<T>::ArrayQueue() {
    max_size_ = DEFAULT_SIZE;
    contents = new T[max_size_];
    begin_ = 0;
    end_ = -1;
    size_ = 0;
}

//! construtor com parametro
template<typename T>
structures::ArrayQueue<T>::ArrayQueue(std::size_t max) {
    max_size_ = max;
    contents = new T[max_size_];
    begin_ = 0;
    end_ = -1;
    size_ = 0;
}

// Destrutor
template<typename T>
structures::ArrayQueue<T>::~ArrayQueue() {
    delete [] contents;
}

// Método para enfileirar - fila circular
template<typename T>
void structures::ArrayQueue<T>::enqueue(const T& data) {
    if (full()) {
        throw std::out_of_range("fila cheia");
    }
    end_ = (end_ + 1) % max_size_;
    contents[end_] = data;
    size_++;
}

//! metodo desenfileirar
template<typename T>
T structures::ArrayQueue<T>::dequeue() {
    if (empty()) {
        throw std::out_of_range("fila vazia");
    }

    T aux = contents[begin_];
    begin_ = (begin_ + 1) % max_size_;
    size_--;

    return aux;
}

//! metodo retorna o ultimo
template<typename T>
T& structures::ArrayQueue<T>::back() {
    if (empty()) {
        throw std::out_of_range("fila vazia");
    }
    return contents[end_];
}

//! metodo limpa a fila
template<typename T>
void structures::ArrayQueue<T>::clear() {
    size_ = 0;
    begin_ = 0;
    end_ = -1;
}

//! metodo retorna tamanho atual
template<typename T>
std::size_t structures::ArrayQueue<T>::size() {
    return size_;
}
//! metodo retorna tamanho maximo
template<typename T>
std::size_t structures::ArrayQueue<T>::max_size() {
    return max_size_;
}

//! metodo verifica se vazio
template<typename T>
bool structures::ArrayQueue<T>::empty() {
    return (size_ == 0);
}

//! metodo verifica se esta cheio
template<typename T>
bool structures::ArrayQueue<T>::full() {
    return (max_size_ == size_);
}
\end{lstlisting}

\end{document}