\appendix{ANEXO 2 - ArrayQueue.h}\\[0.2cm]

\begin{lstlisting}[language=c++]
// Copyright [2024] <LEONARDO DE SOUSA MARQUES>
#include <cstdint>  // std::size_t
#include <stdexcept>  // C++ Exceptions

namespace structures {
template<typename T>
//! classe ArrayQueue
class ArrayQueue {
 public:
    //! construtor padrao
    ArrayQueue();
    //! construtor com parametro
    explicit ArrayQueue(std::size_t max);
    //! destrutor padrao
    ~ArrayQueue();
    //! metodo enfileirar
    void enqueue(const T& data);
    //! metodo desenfileirar
    T dequeue();
    //! metodo retorna o ultimo
    T& back();
    //! metodo limpa a fila
    void clear();
    //! metodo retorna tamanho atual
    std::size_t size();
    //! metodo retorna tamanho maximo
    std::size_t max_size();
    //! metodo verifica se vazio
    bool empty();
    //! metodo verifica se esta cheio
    bool full();

 private:
    T* contents;
    std::size_t size_;  // tamanho atual da fila
    std::size_t max_size_;  // tamanho maximo que a fila pode ter
    int begin_;  // indice do inicio (para fila circular)
    int end_;  // indice do fim (para fila circular)
    static const auto DEFAULT_SIZE = 10u;
};

}  // namespace structures

//! construtor padrao
template<typename T>
structures::ArrayQueue<T>::ArrayQueue() {
    max_size_ = DEFAULT_SIZE;
    contents = new T[max_size_];
    begin_ = 0;
    end_ = -1;
    size_ = 0;
}

//! construtor com parametro
template<typename T>
structures::ArrayQueue<T>::ArrayQueue(std::size_t max) {
    max_size_ = max;
    contents = new T[max_size_];
    begin_ = 0;
    end_ = -1;
    size_ = 0;
}

// Destrutor
template<typename T>
structures::ArrayQueue<T>::~ArrayQueue() {
    delete [] contents;
}

// Método para enfileirar - fila circular
template<typename T>
void structures::ArrayQueue<T>::enqueue(const T& data) {
    if (full()) {
        throw std::out_of_range("fila cheia");
    }
    end_ = (end_ + 1) % max_size_;
    contents[end_] = data;
    size_++;
}

//! metodo desenfileirar
template<typename T>
T structures::ArrayQueue<T>::dequeue() {
    if (empty()) {
        throw std::out_of_range("fila vazia");
    }

    T aux = contents[begin_];
    begin_ = (begin_ + 1) % max_size_;
    size_--;

    return aux;
}

//! metodo retorna o ultimo
template<typename T>
T& structures::ArrayQueue<T>::back() {
    if (empty()) {
        throw std::out_of_range("fila vazia");
    }
    return contents[end_];
}

//! metodo limpa a fila
template<typename T>
void structures::ArrayQueue<T>::clear() {
    size_ = 0;
    begin_ = 0;
    end_ = -1;
}

//! metodo retorna tamanho atual
template<typename T>
std::size_t structures::ArrayQueue<T>::size() {
    return size_;
}
//! metodo retorna tamanho maximo
template<typename T>
std::size_t structures::ArrayQueue<T>::max_size() {
    return max_size_;
}

//! metodo verifica se vazio
template<typename T>
bool structures::ArrayQueue<T>::empty() {
    return (size_ == 0);
}

//! metodo verifica se esta cheio
template<typename T>
bool structures::ArrayQueue<T>::full() {
    return (max_size_ == size_);
}
\end{lstlisting}