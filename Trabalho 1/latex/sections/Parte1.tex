\section{INTRODUÇÃO}

O seguinte relatório tem como objetivo analisar e resolver dois problemas relacionados a um arquivo XML (\textit{Extensible Markup Language}). No primeiro, vamos verificar se o aninhamento das chaves (< e >) e fechamento das marcações (tags) está correto. Já no segundo, vamos calcular a área total que um robô pode limpar partindo de uma posição (x, y) em cada uma das matrizes fornecidas nos arquivos.

\subsection{Modelo de Arquivo}

Antes de começar, vamos entender o modelo do arquivo XML, conforme o bloco abaixo.

\begin{lstlisting}[language=XML]
<cenarios>
<cenario>
<nome>cenario-01</nome>
<dimensoes><altura>20</altura><largura>30</largura></dimensoes>
<robo><x>10</x><y>20</y></robo>
<matriz>
000000000000000000000000000000
000000000000000000000000000000
000000000000000000000000000000
000000000000000000000000000000
000000000000000000000000000000
000000000000000000000000000000
000000000001110000000000000000
000000000011000000000000000000
001100100111110011111000111100
001100100011000011000001110000
001100100011000011111001100000
001100100011000000111101100000
001101100011000010011101110000
000111100011000011111000111100
000000000000000000000000000000
000000000000000000000000000000
000000000000000000000000000000
000000000000000000000000000000
000000000000000000000000000000
000000000000000000000000000000
</matriz>
</cenario>
... % Continue com os outros cenarios da mesma forma
</cenarios>
\end{lstlisting}

Com isso, conseguimos extrair as informações de que o robô começará na posição P, tal que P(x, y) = (10, 20), além de que altura = 20 e largura = 30, por exemplo. Essas informações serão úteis para a questão 2. Em relação aos aninhamentos, percebemos que para o trecho mostrado, ele se demonstra correto. Tendo visto o modelo dos arquivos que estamos trabalhando, podemos dar início a resolução dos problemas apresentados.

\section{PROBLEMA 1}

Para verificar os aninhamentos de uma string que representa todo o arquivo XML lido, é necessário realizar essa conversão. Por isso, o docente forneceu o seguinte bloco de código, na função main, que lê o conteúdo do arquivo com nome passado por linha de comando no terminal - ex: \texttt{cenarios1.xml}.


\begin{lstlisting}[language=c++]
int main() {

    string filename;
    std::cin >> filename;

    // Abertura do arquivo
    ifstream filexml(filename);
    if (!filexml.is_open()) {
        cerr << "Erro ao abrir o arquivo " << filename << endl;
        throw runtime_error("Erro no arquivo XML");
    }

    // Leitura do XML completo para 'texto'
    string texto;
    char character;
    while (filexml.get(character)) {
        texto += character;
    }

    ...
\end{lstlisting}

Tendo a string texto, podemos criar a função \texttt{bool verifica\_aninhamentos(string texto)}, que retorna um valor booleano (verdadeiro ou falso). Para implementar essa função, vamos utilizar a estrutura de dados linear de \textbf{Pilha} (\textit{Array Stack}), que segue a lógica LIFO (\textit{Last in, first out}), que vai armazenar as tags que estão sendo abertas e fechadas. (O código completo da pilha está descrito no Anexo 1.)

Seguindo essa lógica, vamos precisar do comprimento da string para poder instanciar uma pilha com esse tamanho - note que poderíamos utilizar um valor menor, visto que o arquivo não é composto apenas por tags, mas, dessa forma, anunciamos o tamanho máximo. Assim, a lógica para resolver o problema consistiu em percorrer a string texto até o momento em que um caracter '<' é encontrado e, a partir disso, escrever a tag que ele abre até o fechamento da mesma no caractere '>'. O motivo de escrevermos a tag inteira é para caso o caractere seguinte a '<' for uma barra de fechamento '/', pois, dessa forma, garantimos que todas estão sendo devidamente validadas.
\begin{lstlisting}[language=c++][utf-8]
bool verifica_aninhamentos(string texto) {
    // Calcula o comprimento da string de entrada
    int len = texto.length();
    // Cria uma pilha para armazenar as tags de abertura
    structures::ArrayStack<string> stack_tags(len);
    // Inicializa uma string vazia para armazenar o nome da tag atual
    string tag = "";
    // Percorre cada caractere da string de entrada
    for (int i = 0; i < len; i++) {
        // Se o caractere atual for um '<', significa que uma tag vai ser escrita
        if (texto[i] == '<') {
            // Verifica se e uma tag de fechamento se o proximo caractere for '/'
            bool is_closing_tag = (i + 1 < len &&
                                   texto[i + 1] == '/');
            // Se for uma tag de fechamento, avanca o indice para pular o '/'
            if (is_closing_tag) {
                i++;
            }
            // Percorre a string ate encontrar o '>' para capturar o nome da tag
            for (int j = i + 1; j < len; j++) {
                // Se o caractere atual for '>', a tag foi completamente capturada
                if (texto[j] == '>') {
                    // Atualiza 'i' para a posicao de '>', indicando o fim da tag
                    i = j;
                    // Sai do loop interno
                    break;
                }
                // Adiciona o caractere atual ao nome da tag
                tag += texto[j];
            }
            
            // Se for uma tag de fechamento, verifica se a tag de abertura esta no topo da pilha
            if (is_closing_tag) {
                // Se a pilha estiver vazia ou a tag de abertura no topo da pilha nao corresponder a tag de fechamento, retorna falso
                if (stack_tags.empty() || stack_tags.pop() != tag) {
                    return false;
                }
            } else {
                // Se for uma tag de abertura, empilha o nome da tag
                stack_tags.push(tag);
            }
        }
        // Inicializa a variavel tag como vazia para a proxima
        tag = "";
    }
    // Se a pilha estiver vazia, significa que todas as tags foram fechadas
    return stack_tags.empty();
}
\end{lstlisting}

\subsection{Exemplo de Funcionamento}
Vamos observar o exemplo da verificação de aninhamentos, com as tags sendo empilhadas e desempilhadas à medida que são analisadas. Considere o código XML:
\begin{lstlisting}[language=XML]
<cenarios>
  <cenario>
    <nome>cenario-01</nome>
  </cenario>
</cenarios>
\end{lstlisting}

Com isso, podemos esquematizar a tabela que representa as etapas de empilhamento para verificar os passos dos algoritmo.

\newpage

\begin{table}[h!]
\centering
\begin{tabular}{|c|c|}
\hline
\textbf{Etapa} & \textbf{Pilha} \\
\hline
1. Início & \(\emptyset\) \\
\hline
2. Tag \texttt{<cenarios>} encontrada & \texttt{<cenarios>} \\
\hline
3. Tag \texttt{<cenario>} encontrada & \texttt{<cenario>} \\
                                      & \texttt{<cenarios>} \\
\hline
4. Tag \texttt{<nome>} encontrada & \texttt{<nome>} \\
                                   & \texttt{<cenario>} \\
                                   & \texttt{<cenarios>} \\
\hline
5. Tag \texttt{</nome>} encontrada e removida & \texttt{<cenario>} \\
                                              & \texttt{<cenarios>} \\
\hline
6. Tag \texttt{</cenario>} encontrada e removida & \texttt{<cenarios>} \\
\hline
7. Tag \texttt{</cenarios>} encontrada e removida & \(\emptyset\) \\
\hline
\end{tabular}
\caption{Exemplo de verificação de aninhamento}
\end{table}

A pilha está vazia no final (\(\emptyset\)), o que indica que todas as tags de abertura tiveram uma correspondente tag de fechamento, confirmando que o aninhamento está correto.

Também podemos verificar um caso em que tenhamos erros de aninhamento. 

\begin{lstlisting}[language=XML]
<cenarios>
  <cenario>
    <nome>cenario-01<nome> -- erro;
  </cenario>
</cenarios>
\end{lstlisting}

\begin{table}[h!]
\centering
\begin{tabular}{|c|c|}
\hline
\textbf{Etapa} & \textbf{Estado da Pilha} \\
\hline
1. Início & \(\emptyset\) \\
\hline
2. Tag \texttt{<cenarios>} encontrada & \texttt{<cenarios>} \\
\hline
3. Tag \texttt{<cenario>} encontrada & \texttt{<cenario>} \\
                                      & \texttt{<cenarios>} \\
\hline
4. Tag \texttt{<nome>} encontrada & \texttt{<nome>} \\
                                        & \texttt{<cenario>} \\
                                        & \texttt{<cenarios>} \\
\hline
5. Tag \texttt{<nome>} encontrada & \texttt{<nome>} \\
                                     & \texttt{<nome>} \\
                                     & \texttt{<cenario>} \\
                                     & \texttt{<cenarios>} \\
\hline
6. Tag \texttt{</cenario>} encontrada & \texttt{<nome>} \\
                                     & \texttt{<nome>} \\
                                     & \texttt{<dimensoes>} \\
                                     & \texttt{<cenario>} \\
                                     & \texttt{<cenarios>} \\
\hline
\end{tabular}
\caption{Processo de Verificação com Erro no Aninhamento}
\end{table}

No passo 6 o método irá retornar $false$, pois \texttt{<cenario>} não está no topo da pilha, visto que houve um erro durante o processo. Nesse caso, como a tag \texttt{<nome>} não foi fechada, ela impede o fechamento das anteriores.


