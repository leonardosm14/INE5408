\section{ARQUIVO MAIN}

Tendo criado os métodos que resolvem os problemas, precisamos chamá-los na função main(). É importante ressaltar que para o segundo problema, como podemos ter diversas matrizes no mesmo XML, é necessário fazer um loop que instância um $Cenario$ com posição inicial referente à posição final do último cenário lido. Desse modo, teremos a função main completa.

\begin{lstlisting}[language=c++]
int main() {

    string filename;
    cin >> filename;

    ifstream filexml(filename);
    if (!filexml.is_open()) {
        cerr << "Erro ao abrir o arquivo " << filename << endl;
        throw runtime_error("Erro no arquivo XML");
    }

    // Leitura do XML completo para 'texto'
    string texto;
    char character;
    while (filexml.get(character)) {
        texto += character;
    }

    // Problema 1
    bool aninhamentos = verifica_aninhamentos(texto);

    if (!aninhamentos) {
        cout << "Erro de aninhamentos" << endl;
    } else {
        // Problema 2
         size_t num_matrizes = quantidade_matrizes(texto);
         size_t i = 0;
         for (size_t m = 0; m < num_matrizes; m++) {
            Cenario c(texto, i); 
            size_t area = calcula_area(c);
            cout << c.nome << " " << area << endl;
            i = c.indice_final;
        }
    }
    return 0;
}
\end{lstlisting}